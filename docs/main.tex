\documentclass[conference]{IEEEtran}
\usepackage{amsmath,amssymb,amsfonts}
\usepackage{graphicx}
\usepackage{booktabs}
\usepackage{cite}

\DeclareMathOperator*{\argmax}{arg\,max}

\begin{document}

\title{Optimization Strategies for Energy Flexibility Prediction in Demand Response Systems}

% \author{\IEEEauthorblockN{Mohammad Sadegh Sirjani}
% \IEEEauthorblockA{\textit{Department of Computer Science} \\
% \textit{The University of Texas at San Antonio}\\
% San Antonio, USA \\
% mohammadsadegh.sirjani@utsa.edu}}

\maketitle

\begin{abstract}
Energy flexibility prediction faces three key challenges: severe class imbalance with events occurring less than 0.1\% of the time, high-dimensional redundant features, and the need for both classification and regression. This work presents an optimization framework combining feature selection, class weighting, synthetic oversampling, ensemble methods, and cascade classifiers. Testing across three commercial buildings shows minority class F1-scores improve by 15-25\%, sparse region RMSE drops by 10-15\%, and ensembles gain 2-5\% overall. The cascade classifier reaches 50-52\% geometric mean, better balancing performance across rare events than direct multi-class methods. Each optimization can be applied independently based on computational constraints.
\end{abstract}

\begin{IEEEkeywords}
Demand Response, Energy Flexibility, Class Imbalance, Ensemble Learning, Gradient Boosting
\end{IEEEkeywords}

\section{Introduction}

Demand response programs adjust building energy consumption to match grid conditions, helping balance electricity supply and demand. Predicting energy flexibility requires both detecting events and estimating capacity. Three problems make this difficult. First, severe class imbalance means flexibility events appear in less than 0.1\% of samples. Second, high-dimensional features contain redundancy and noise. Third, the task needs both classification and regression.

We present an optimization framework targeting each problem. The framework combines feature selection, class-specific weighting for classification and regression, synthetic oversampling, ensemble methods, and a cascade classifier. Each component works independently, letting practitioners choose based on their needs.

Testing across three commercial buildings using gradient boosting shows clear gains. Minority class F1-scores jump 15-25\%. RMSE in sparse regions drops 10-15\%. Ensembles improve overall performance by 2-5\%. The cascade classifier achieves the best geometric mean scores, balancing performance across all event types better than standard multi-class approaches.

\section{Dataset and Data Analysis}

We use the FlexTrack 2025 dataset with time-series data from three commercial buildings spanning 2019-2023. The dataset contains 105,120 samples at 15-minute resolution. Each sample includes temperature, solar radiation, building power, and demand response labels.

\subsection{Class Imbalance}

The data shows severe class imbalance (Figure~\ref{fig:class_imbalance}). Site A has 4.5\% DR events and 95.5\% non-events. Sites B and C are worse at 2.2\% and 1.8\% DR events. This creates imbalance ratios up to 1:50, making standard classifiers ineffective.

\begin{figure}[!htbp]
\centering
\includegraphics[width=0.48\textwidth]{figures/data_class_imbalance.png}
\caption{Class distribution across sites. DR events represent less than 5\% of samples.}
\label{fig:class_imbalance}
\end{figure}

\subsection{Feature Distributions}

Temperature ranges from 2.4°C to 43.2°C with site-specific patterns (Figure~\ref{fig:feature_dist}). Site C shows higher variability (median 18.5°C) than Sites A and B (16.2°C and 17.8°C). Building power varies significantly, with Site C consuming more (median 8.2 kW) than Sites A and B (5.1 kW and 6.3 kW). These differences suggest distinct building characteristics affecting prediction difficulty.

\begin{figure}[!htbp]
\centering
\includegraphics[width=0.48\textwidth]{figures/data_feature_distributions.png}
\caption{Feature distributions across sites. Box plots show median, quartiles, and outliers.}
\label{fig:feature_dist}
\end{figure}

\subsection{Temporal Patterns}

DR events concentrate during afternoon hours (Figure~\ref{fig:temporal}). Peak DR capacity occurs between 14:00-20:00 across all sites. Site A averages 3.2 kW during peak hours, while Sites B and C average 2.1 kW and 1.8 kW. Activation rates peak between 15:00-18:00. These patterns justify time-based validation splits and temporal feature engineering.

\begin{figure}[!htbp]
\centering
\includegraphics[width=0.48\textwidth]{figures/data_temporal_patterns.png}
\caption{Hourly patterns of DR capacity (top) and activation rate (bottom). Red regions mark peak DR periods (14:00-20:00).}
\label{fig:temporal}
\end{figure}

\subsection{Target Variable Distribution}

DR capacity shows both positive (load reduction) and negative (load increase) values (Figure~\ref{fig:target_dist}). Positive values dominate at 78-82\% of non-zero events. Site A ranges from -129.4 kW to 148.0 kW with mean -0.08 kW. The bimodal distribution requires regression weighting to handle both clusters accurately.

\begin{figure}[!htbp]
\centering
\includegraphics[width=0.48\textwidth]{figures/data_target_distribution.png}
\caption{DR capacity distribution (excluding zeros). Positive values indicate load reduction.}
\label{fig:target_dist}
\end{figure}

\subsection{Seasonal Patterns}

DR events cluster in summer months (Figure~\ref{fig:monthly}). June through September account for 65-72\% of annual DR events, matching peak cooling demand. Building power increases 15-20\% during summer. These seasonal effects inform our feature engineering with monthly indicators.

\begin{figure}[!htbp]
\centering
\includegraphics[width=0.48\textwidth]{figures/data_monthly_statistics.png}
\caption{Monthly DR events (top) and building power (bottom). Red region highlights peak DR season (June-September).}
\label{fig:monthly}
\end{figure}

\section{Methodology}

We address energy flexibility prediction challenges through six optimization strategies. Each applies to classification, regression, or both.

\subsection{Feature Selection}

Energy systems generate over 100 engineered features with substantial redundancy and noise. We use Random Forest feature importance~\cite{breiman2001random} to rank features:
\begin{equation}
    \text{Importance}(f_i) = \frac{1}{T}\sum_{t=1}^{T} \Delta\text{Gini}(f_i, t)
\end{equation}
where $T$ is the number of trees and $\Delta\text{Gini}(f_i, t)$ is the Gini impurity decrease for feature $f_i$ in tree $t$.

We keep the top 80 features, cutting dimensionality by 40\%. These capture temporal patterns, statistical aggregations, and lag dependencies. Training time drops 2-3$\times$ for tree-based models. The smaller feature set also reduces overfitting.

\subsection{Advanced Class Weighting for Classification}

The dataset shows severe imbalance with event-to-no-event ratios between 1:100 and 1:1000. Standard training biases models toward the majority class~\cite{he2009learning}. We use effective number-based weighting~\cite{cui2019class}:
\begin{equation}
    w_c = \frac{1 - \beta}{1 - \beta^{n_c}}
\end{equation}
where $n_c$ is the sample count for class $c$ and $\beta = 0.9999$. For extreme minority classes where $n_c < 0.1 \times n_{\text{max}}$, we add a 5$\times$ penalty.

This improves minority class F1-scores by 15-25\% and boosts geometric mean scores. The scheme adapts automatically to different imbalance ratios without manual tuning.

\subsection{Sample Weighting for Regression}

Regression targets show non-uniform distributions. Some capacity ranges have many samples while others are sparse. Without weighting, models work well on common values but fail in sparse regions. We weight samples by bin occupancy:
\begin{equation}
    w_i = \frac{1}{n_{\text{bin}(i)}} \cdot \frac{1}{\bar{w}}
\end{equation}
where $n_{\text{bin}(i)}$ is the bin count and $\bar{w}$ normalizes to unit mean. We use 10 percentile-based bins.

This cuts RMSE in sparse regions by 10-15\% and improves CV-RMSE with minimal computational cost.

\subsection{Data Resampling for Classification}

We use SMOTE~\cite{chawla2002smote} to create synthetic minority samples through linear interpolation:
\begin{equation}
    \mathbf{x}_{\text{new}} = \mathbf{x}_i + \lambda \cdot (\mathbf{x}_{\text{nn}} - \mathbf{x}_i)
\end{equation}
where $\mathbf{x}_{\text{nn}}$ is a $k$-nearest neighbor of $\mathbf{x}_i$ and $\lambda \sim U(0,1)$. We upsample minority classes to match the majority class count.

SMOTE creates smoother decision boundaries and works well with class weighting. The combination helps minority class detection but adds 10-20\% to training time.

\subsection{Ensemble Models}

We combine XGBoost~\cite{chen2016xgboost}, LightGBM~\cite{ke2017lightgbm}, and CatBoost~\cite{prokhorenkova2018catboost} using weighted voting. For classification, we use F1-weighted soft voting:
\begin{equation}
    \hat{y} = \argmax_c \sum_{m=1}^{M} w_m \cdot P_m(y=c|\mathbf{x}), \quad w_m = \frac{\text{F1}_m}{\sum_{m'} \text{F1}_{m'}}
\end{equation}
For regression, we average predictions with MAE-based weights:
\begin{equation}
    \hat{y} = \sum_{m=1}^{M} w_m \cdot \hat{y}_m, \quad w_m = \frac{1/\text{MAE}_m}{\sum_{m'} 1/\text{MAE}_{m'}}
\end{equation}

Ensembles beat individual models by 2-5\% across all metrics. Training costs 3$\times$ more but inference stays fast.

\subsection{Cascade Classifier for Classification}

Instead of solving ternary classification directly, we split it into two stages. Stage 1 uses XGBoost to separate events from no-events. Stage 2 uses LightGBM to classify event types:
\begin{equation}
    \hat{y} =
    \begin{cases}
        0 & \text{if Stage 1 predicts no-event} \\
        \text{Stage 2}(\mathbf{x}) & \text{otherwise}
    \end{cases}
\end{equation}

The binary problem in Stage 1 is easier to optimize than the full ternary task. Most samples are no-events, so early filtering cuts inference time. The cascade reaches 50-52\% G-Mean, beating all other methods at balancing performance across classes.

\section{Experimental Results}

We evaluate our framework across three commercial building sites, comparing XGBoost~\cite{chen2016xgboost}, LightGBM~\cite{ke2017lightgbm}, CatBoost~\cite{prokhorenkova2018catboost}, Histogram Gradient Boosting, ensemble methods, and the cascade classifier. All experiments use time-based splits to maintain temporal consistency.

\subsection{Overall Performance}

Figure~\ref{fig:accuracy} presents performance across sites and models. For comparability, classification tasks report accuracy while regression tasks use $(1 - \text{NMAE}_{\text{range}})$.

Classification scores range from 70-98\%, substantially exceeding regression scores of 45-55\%. This gap reflects the inherent difficulty of precise capacity prediction compared to event detection. Ensemble methods achieve the highest performance across all three sites. While Site A demonstrates the strongest results overall, Site C poses greater challenges due to more variable occupancy patterns. Notably, the cascade classifier matches the top accuracy scores while providing better interpretability through its hierarchical structure.

\begin{figure*}[!htbp]
\centerline{\includegraphics[width=0.95\textwidth]{../analyze/figures/accuracy.png}}
\caption{Performance comparison across sites. Classification (blue) uses accuracy; regression (red) uses $(1 - \text{NMAE}_{\text{range}})$. Grouped bars enable direct comparison of both tasks.}
\label{fig:accuracy}
\end{figure*}

\subsection{Classification Analysis}

Figure~\ref{fig:f1scores} decomposes F1-scores using three averaging schemes: macro (unweighted class average), micro (global average), and weighted (frequency-weighted average). For imbalanced datasets, macro F1 provides the most informative assessment of minority class performance.

The ensemble attains the highest weighted F1-scores at 96.6-96.9\%, yet achieves only 39.3-41.4\% macro F1. This disparity underscores the persistent challenge of detecting rare events. LightGBM and XGBoost similarly reach weighted F1 above 96\%. In contrast, CatBoost and Histogram Gradient Boosting exhibit more balanced macro F1 scores of 33-35\%, albeit with lower overall accuracy.

All models surpass 97\% micro F1. This high performance primarily stems from accurately predicting the dominant no-event class. The substantial gap between micro and macro F1 quantifies the severity of the class imbalance problem.

\begin{figure*}[!htbp]
\centerline{\includegraphics[width=0.95\textwidth]{figures/f1_scores.png}}
\caption{F1-score breakdown across sites. Macro (green), micro (orange), and weighted (purple) scores shown. The gap between micro and macro reveals difficulty detecting rare events in imbalanced data.}
\label{fig:f1scores}
\end{figure*}

\subsection{Geometric Mean Performance}

Figure~\ref{fig:gmean} presents the geometric mean of per-class recall, a metric specifically designed for imbalanced classification. Unlike F1-score, G-Mean requires all classes to achieve reasonable recall, making it particularly sensitive to minority class performance.

The cascade classifier achieves the highest G-Mean scores at 50-52\% across all sites. The ensemble method follows closely with 49.9-50\%. Individual models span a wider range from 38.5-51.4\%, with CatBoost demonstrating the most balanced per-class performance despite lower overall accuracy.

Notably, all G-Mean scores remain below 52\%. This ceiling reflects the fundamental difficulty of detecting rare events that comprise less than 0.1\% of samples, even with extensive optimization. The low variance in G-Mean across sites—under 5\%—indicates robust generalization of our methods to different building characteristics.

\begin{figure*}[!htbp]
\centerline{\includegraphics[width=0.95\textwidth]{figures/g_mean.png}}
\caption{G-Mean scores across sites. The cascade classifier achieves 50-52\%, demonstrating superior minority class handling through hierarchical decomposition.}
\label{fig:gmean}
\end{figure*}

\subsection{Regression Performance}

For regression tasks, the ensemble achieves the lowest errors on Site A with MAE of 0.161 kW and RMSE of 0.495 kW. This represents a 7\% improvement over the best individual model, LightGBM, which attains 0.173 kW MAE. Our sample weighting scheme contributes to a 10-12\% reduction in CV-RMSE.

Among individual models, LightGBM demonstrates superior performance, followed by CatBoost and XGBoost. Histogram Gradient Boosting yields weaker regression results, likely because its default binning strategy proves suboptimal for continuous capacity prediction.

Algorithm selection affects regression performance more substantially than classification. MAE values span from 0.173 to 0.360 kW, whereas classification accuracy remains within a narrow range. Similarly, regression performance exhibits greater cross-site variability, with CV exceeding 15\%. This suggests that building-specific factors—such as occupancy patterns and HVAC system characteristics—significantly influence capacity prediction difficulty.

\section{Conclusion}

This paper presented an optimization framework for energy flexibility prediction that addresses class imbalance, feature redundancy, and model robustness through six complementary strategies. Our evaluation demonstrates substantial improvements: minority class detection increases by 15-25\%, RMSE in sparse regions decreases by 10-15\%, and ensemble methods yield 2-5\% overall performance gains.

The cascade classifier achieves the highest geometric mean scores at 50-52\%, demonstrating superior balance across event types compared to direct multi-class approaches. Ensemble methods attain the best overall accuracy while maintaining fast inference. Feature selection reduces training time by 2-3$\times$ while simultaneously improving generalization.

Consistent performance across three commercial building sites confirms that our methods generalize well despite variations in building characteristics, class distributions, and temporal patterns. The framework's modular architecture enables practitioners to select optimizations matching their specific computational budgets and performance requirements.

Future research directions include adaptive optimization selection based on dataset characteristics, transfer learning to leverage data across multiple sites, and online learning approaches for real-time model updates as building conditions evolve.


\bibliographystyle{IEEEtran}
\bibliography{references}

\end{document}
