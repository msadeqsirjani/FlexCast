\section{Introduction}

Demand response programs enable buildings to adjust their energy consumption in response to grid conditions, playing an increasingly critical role in balancing electricity supply and demand. Accurate prediction of energy flexibility—encompassing both event detection and capacity estimation—remains difficult due to several interrelated challenges. The data exhibits severe class imbalance, with flexibility events occurring in less than 0.1\% of observations. High-dimensional feature spaces introduce redundancy and noise, while the problem itself requires both classification (detecting events) and regression (estimating capacity).

This paper presents an optimization framework that systematically addresses these challenges. We integrate feature selection to reduce dimensionality, advanced weighting schemes tailored separately for classification and regression, synthetic oversampling for minority classes, ensemble methods, and a cascade classifier architecture. Unlike monolithic approaches, our framework allows practitioners to selectively apply optimizations based on their computational constraints and performance requirements.

We evaluate our methods across three commercial building sites using gradient boosting models. Results demonstrate substantial improvements: minority class F1-scores increase by 15-25\%, RMSE in sparse regions decreases by 10-15\%, and ensemble methods achieve 2-5\% overall performance gains. The cascade classifier attains the highest geometric mean scores, indicating better balance across event types.
