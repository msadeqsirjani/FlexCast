\section{Introduction}

Demand response programs adjust building energy consumption to match grid conditions, helping balance electricity supply and demand. Predicting energy flexibility requires both detecting events and estimating capacity. Three problems make this difficult. First, severe class imbalance means flexibility events appear in less than 0.1\% of samples. Second, high-dimensional features contain redundancy and noise. Third, the task needs both classification and regression.

We present an optimization framework targeting each problem. The framework combines feature selection, class-specific weighting for classification and regression, synthetic oversampling, ensemble methods, and a cascade classifier. Each component works independently, letting practitioners choose based on their needs.

Testing across three commercial buildings using gradient boosting shows clear gains. Minority class F1-scores jump 15-25\%. RMSE in sparse regions drops 10-15\%. Ensembles improve overall performance by 2-5\%. The cascade classifier achieves the best geometric mean scores, balancing performance across all event types better than standard multi-class approaches.
