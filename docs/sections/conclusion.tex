\section{Conclusion}

This paper presented an optimization framework for energy flexibility prediction that addresses class imbalance, feature redundancy, and model robustness through six complementary strategies. Our evaluation demonstrates substantial improvements: minority class detection increases by 15-25\%, RMSE in sparse regions decreases by 10-15\%, and ensemble methods yield 2-5\% overall performance gains.

The cascade classifier achieves the highest geometric mean scores at 50-52\%, demonstrating superior balance across event types compared to direct multi-class approaches. Ensemble methods attain the best overall accuracy while maintaining fast inference. Feature selection reduces training time by 2-3$\times$ while simultaneously improving generalization.

Consistent performance across three commercial building sites confirms that our methods generalize well despite variations in building characteristics, class distributions, and temporal patterns. The framework's modular architecture enables practitioners to select optimizations matching their specific computational budgets and performance requirements.

Future research directions include adaptive optimization selection based on dataset characteristics, transfer learning to leverage data across multiple sites, and online learning approaches for real-time model updates as building conditions evolve.
