\section{Dataset and Data Analysis}

We use the FlexTrack 2025 dataset with time-series data from three commercial buildings spanning 2019-2023. The dataset contains 105,120 samples at 15-minute resolution. Each sample includes temperature, solar radiation, building power, and demand response labels.

\subsection{Class Imbalance}

The data shows severe class imbalance (Figure~\ref{fig:class_imbalance}). Site A has 4.5\% DR events and 95.5\% non-events. Sites B and C are worse at 2.2\% and 1.8\% DR events. This creates imbalance ratios up to 1:50, making standard classifiers ineffective.

\begin{figure}[!htbp]
\centering
\includegraphics[width=0.48\textwidth]{figures/data_class_imbalance.png}
\caption{Class distribution across sites. DR events represent less than 5\% of samples.}
\label{fig:class_imbalance}
\end{figure}

\subsection{Feature Distributions}

Temperature ranges from 2.4°C to 43.2°C with site-specific patterns (Figure~\ref{fig:feature_dist}). Site C shows higher variability (median 18.5°C) than Sites A and B (16.2°C and 17.8°C). Building power varies significantly, with Site C consuming more (median 8.2 kW) than Sites A and B (5.1 kW and 6.3 kW). These differences suggest distinct building characteristics affecting prediction difficulty.

\begin{figure}[!htbp]
\centering
\includegraphics[width=0.48\textwidth]{figures/data_feature_distributions.png}
\caption{Feature distributions across sites. Box plots show median, quartiles, and outliers.}
\label{fig:feature_dist}
\end{figure}

\subsection{Temporal Patterns}

DR events concentrate during afternoon hours (Figure~\ref{fig:temporal}). Peak DR capacity occurs between 14:00-20:00 across all sites. Site A averages 3.2 kW during peak hours, while Sites B and C average 2.1 kW and 1.8 kW. Activation rates peak between 15:00-18:00. These patterns justify time-based validation splits and temporal feature engineering.

\begin{figure}[!htbp]
\centering
\includegraphics[width=0.48\textwidth]{figures/data_temporal_patterns.png}
\caption{Hourly patterns of DR capacity (top) and activation rate (bottom). Red regions mark peak DR periods (14:00-20:00).}
\label{fig:temporal}
\end{figure}

\subsection{Target Variable Distribution}

DR capacity shows both positive (load reduction) and negative (load increase) values (Figure~\ref{fig:target_dist}). Positive values dominate at 78-82\% of non-zero events. Site A ranges from -129.4 kW to 148.0 kW with mean -0.08 kW. The bimodal distribution requires regression weighting to handle both clusters accurately.

\begin{figure}[!htbp]
\centering
\includegraphics[width=0.48\textwidth]{figures/data_target_distribution.png}
\caption{DR capacity distribution (excluding zeros). Positive values indicate load reduction.}
\label{fig:target_dist}
\end{figure}

\subsection{Seasonal Patterns}

DR events cluster in summer months (Figure~\ref{fig:monthly}). June through September account for 65-72\% of annual DR events, matching peak cooling demand. Building power increases 15-20\% during summer. These seasonal effects inform our feature engineering with monthly indicators.

\begin{figure}[!htbp]
\centering
\includegraphics[width=0.48\textwidth]{figures/data_monthly_statistics.png}
\caption{Monthly DR events (top) and building power (bottom). Red region highlights peak DR season (June-September).}
\label{fig:monthly}
\end{figure}
